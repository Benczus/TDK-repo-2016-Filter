\chapter{Introduction.}
\label{chap:introduction}

Providing a filter to accurately predict and remove noise from measurements is a very active research area.
The Horus system showed that using time series instead of single values increases efficiency.
Filters can be applied as preprocessing algorithm to a WLAN to make steady connection to an Access Point.
Analysis of time series of WiFi RSSI measurements could lead to the development of an efficient client site filtering method for indoor positioning systems.
There are various time series filtering methods in the literature. This paper focuses only on two simple solution which are based on time windowing.
In the case that the two filters work efficiently, the process could be applied to networks with different filters to reach an optimal solution.
Computational complexity also gives an important constrain of these filtering methods, because the client devices usually have limited computational capacity and battery life.

\section{ILONA}
The presented results are connected to the Indoor Positioning Research at the University of Miskolc. The ILONA System is web application for indoor positioning. It provides positioning functions for client. 
ILONA stands foor  INdoor LOcation and NAvigation which is a web application
created to perform indoor positioning and navigation tasks.
It is made up by loosely coupled components such as \texttt{measurement, positioning, navigation} and \texttt{tracking}.
This paper focuses on the data analysis and data mining of the ILONA System.
A proper client side filtering method could increase the performance and the accuracy of the positioning service of the ILONA System.

