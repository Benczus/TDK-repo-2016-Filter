\chapter{Evaluation}
\label{chap:evaluation}
\section{Test Environment}
The measurements were recorded in the University of Miskolc's Institution of Information Science, Department of Information Technology.The recorded values were taken with 4 identical smart phones with Android operating system. Two of the tests were recorded without any disturbance, just the natural reflective surfaces in the room, and two were recorded with people present in the room. The latest showed a lot of inconsistencies on the recorded values, further increasing the need for a filter on the preprocessed data. As noise grows, the filters value increases, as it can be smoothed out the naturally occurring  jumps on the diagram.
The following sample is taken from the data set used to test the filters on:
\begin{figure}[h!]
\begin{verbatim}
Time,SSID,MAC,Signal
2016/02/02.09:43:33,IITAP1,E0:5F:B9:0C:71:27,-39
2016/02/02.09:43:33,IITAP1-GUEST,E2:5F:B9:0C:71:27,-39
2016/02/02.09:43:33,KRZ,0:18:E7:DE:A3:90,-48
2016/02/02.09:43:33,LABOR,00:14:C1:33:A0:78,-72
2016/02/02.09:43:33,GEIAKFSZ,F8:66:F2:AD:E6:91,-76
2016/02/02.09:43:33,doa207,14:CC:20:57:3D:0C,-67
2016/02/02.09:43:33,dd,00:19:E0:65:E4:F2,-84
2016/02/02.09:43:33,doa208,F8:66:F2:AD:E6:79,-81
2016/02/02.09:43:35,IITAP1,E0:5F:B9:0C:71:27,-40

\end{verbatim}
\caption{Data set sample}
\label{fig:Datasample}
\end{figure}

\section{Comparison}
The filters have clearly shown that their usage have increased the accuracy and steadiness of the WiFi signals. These results mean that the usage of preprocessing filters can be used to steadily improve positioning using the Horus System.



The comparison of the diagrams:
\begin{figure}[h!]
	\centering
		\includegraphics[width=.9\linewidth]{figures/comp1.png}
		\caption{Comparison of the values of the raw and the values after first filter \cite{firstcomp}}\label{fig:firstComp}
\end{figure}

\begin{figure}[h!]
	\centering
		\includegraphics[width=.9\linewidth]{figures/comp2.png}
		\caption{Comparison of the values of the raw and the values after second filter \cite{secondcomp}}\label{fig:secondComp}
\end{figure}
\begin{figure}[h!]
	\centering
		\includegraphics[width=.9\linewidth]{figures/comp3.png}
		\caption{Comparison of the values of the two filters \cite{thirdcomp}}\label{fig:thirdComp}
\end{figure}


easy to understand, steady connection.
\section{Suggestion} 
Way ahead of it's time, probably will be used in the future, and can be usedin great effect in theoretical envirment. 
